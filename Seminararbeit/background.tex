\chapter{Hintergr�nde}
\section{Einleitung}
Bei \textit{"`The right to be forgotten"'}\cite{ward} handelt es sich um einen Pr�zedenzfall am Europ�ischen Gerichtshof\footnote[1]{http://europa.eu/about-eu/institutions-bodies/court-justice/index\_de.htm}, welcher am 13. Mai 2014 sein Urteil dazu traf. Dabei ging es darum, dass Google\footnote[2]{https://www.google.com/} Links die Privatsph�re von Einzelnen verletzen k�nnen. Der Europ�ische Gerichtshof erlie� daraufhin das Recht auf L�schung von Suchresultaten von Suchmaschinen im Internet und versuchte eine Balance zwischen privaten und �ffentlichen Interessen zu wahren.

\section{Der Pr�zedenzfall}
Der spanischer Anwalt Mario Costeja Gonz�les\cite{wagner,bygrave} hatte im Jahr 1998 Sozialversicherungsschulden und musste daher Grundbesitz auf einer Auktion versteigern. Zu dieser Aff�re gab es zwei Artikel in der Tageszeitung "`La Vanguardia Ediciones"'\footnote[3]{http://hemeroteca.lavanguardia.com/preview/1998/03/09/pagina-13/33837533/pdf.html}\footnote[4]{http://hemeroteca.lavanguardia.com/preview/1998/03/09/pagina-23/33842001/pdf.html}. Nachdem Herr Gonz�les seine finanziellen Probleme bereinigte und alle Registereintr�ge zu diesem Vorfall verfallen sind, klagte dieser im Jahr 2010 die Zeitung und forderte, dass sie die ihm betreffenden Inhalte entfernen, da die Artikel immer noch im Internet lesbar und auffindbar waren. 

Die Zeitung weigerte sich dem nachzukommen und berief sich auf Presse- und Meinungsfreiheit von Presse und Zeitungen. Daraufhin klagte Herr Gonz�les gegen Google, um die zwei Suchergebnisse zu l�schen, welche erschienen, sobald man seinen Namen auf Google suchte. Auch Google weigerte sich die Suchresultate zu l�schen, wodurch der Prozess bis zum Europ�ischen Gerichtshof getragen wurde.

\section{Urteil}
Der Europ�ische Gerichtshof erlie� daraufhin das Urteil\cite{ward}, dass man ein Recht auf L�schung von Suchergebnissen von Suchmaschinen im Internet hat, wenn die verlinkten Daten
\begin{itemize}
	\item unangemessen
	\item irrelevant
	\item exzessiv
	\item nicht mehr aktuell
	\item nicht mehr relevant
\end{itemize}
sind und nicht in Konflikt mit dem �ffentlichen Interesse stehen.

Von dem Urteil sind alle Suchmaschinen im Internet betroffen, da der Europ�ische Gerichtshof die Suchmaschinen nicht nur als Pr�sentatoren von Daten im Internet, sondern auch als Data Controller einstufte und diese daher in eine andere Gerichtsbarkeit fallen. Nat�rlich beruft sich das Urteil nur auf den europ�ischen Raum.