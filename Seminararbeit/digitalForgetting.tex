\chapter{Digitales Vergessen}
\section{Vergeben und Vergessen}
Das Recht auf Vergessen ist keine Erfindung des Europ�ischen Gerichtshof, sondern viel mehr eine seit Jahrhunderten angewandte Praxis, welche verschieden Hintergr�nde haben kann. Ein Jeder kennt das kleine Sprichwort \textit{"`Vergeben und Vergessen"'}. Es gibt schon immer religi�se oder kulturelle Gepflogenheiten, um einem Mitmenschen zu erm�glichen, dass seine Taten vergeben werden k�nnen. Und vergebene Taten werden in der Regel auch vergessen.\cite{bishop}

Falls manche Taten nicht vergeben und somit auch nicht vergessen werden k�nnen, gab es fr�her immer noch die M�glichkeit einen Neustart in einer neuen Umgebung zu machen und so seinen S�nden zu entfliehen. Heutzutage sind Fehltritte meist nur eine Googlesuche entfernt und ein Neustart wird beinahe Unm�glich. Daher sollte man dar�ber nachdenken, ob nicht auch digitale S�nden vergeben werden k�nnen und somit auch vergessen.

\section{Digitale Identit�t}
Das Internet bietet die M�glichkeit sich neu zu erfinden und mehrere Pseudonyme anzunehmen. Dies wurde anfangs noch gef�rdert, indem man sich zum Beispiel verschiedene Alias in Foren anlegen konnte. Heutzutage ist es mit den Sozialen Medien nicht mehr so einfach verschiedene digitale Identit�ten zu betreiben, da es zusehends zu einer Verschmelzung von privaten und �ffentlichen Leben kommt. Au�erdem sollte man im Internet immer im Hinterkopf behalten, dass nicht immer alles stimmen muss, was man zu lesen bekommt. Das Internet wird immer mehr als L�gennetz missbraucht.

\section{M�glichkeiten}
In den n�chsten Abs�tzen wird darauf eingegangen welche grundlegenden M�glichkeiten und Methoden es gibt, um Daten verschwinden zu lassen.
\subsection{Verbreitung kontrollieren}
Ein Ansatz ist, dass man die Verbreitung seiner Daten im Internet von Anfang an kontrolliert. Eine M�glichkeit daf�r ist der Originator-Controlled Access Control\cite{graubert} womit man als Ersteller der Daten anderen Zugang gew�hren und wieder verbieten kann. Eine weitere Idee ist, dass man nur Links zu seinen Daten verbreitet und nicht die Daten selbst. So kann man die Daten jederzeit wieder vom Netz nehmen. Ein anderer Vorgang bietet an, dass man allen Daten eine gewisse Lebenszeit mitgibt, sodass die Daten und alle Kopien davon nach einer gewissen Zeit nicht mehr zur Verf�gung stehen.

Hier stellt sich jedoch die Frage, wer die Daten besitzt. Nur der Ersteller beziehungsweise Besitzer der Daten kann die Verbreitung kontrollieren.

\subsection{Verstecken}
Eine weitere Ann�herung seine Daten zu sch�tzen ist, dass man sie durch Irref�hren und �berfluten versteckt. Beispielsweise wurde Deutschland im zweiten Weltkrieg get�uscht, indem die Alliierten haufenweise falscher Daten produzierten und lieferten, dass sie �ber Sardinien einfallen, obwohl das Angriffsziel Sizilien war. Ein weiteres bekanntes Beispiel daf�r sind Honeypods, die dazu genutzt werden, einen potenziellen Angreifer von den wirklichen Daten abzulenken und ihm ein anderes Ziel stattdessen anzubieten.

Man kann auch versuchen unerw�nschte Daten �ber sich selbst jemand anderes unterzuschieben. So bleiben die Daten in Takt, jedoch wird das Ziel der Informationen ge�ndert. Diese Herangehensweise ist ethnisch sehr fragw�rdig.

