\chapter{Reaktionen}
\section{Google}
Google wehrte sich anfangs mit der Begr�ndung, dass sie keine Daten im Internet kontrollieren sondern lediglich lokalisieren und pr�sentieren. Au�erdem warnten sie vor einer Balkanisierung des Internet, wenn sich Europa eigenst�ndig machen m�chte. Der Europ�ische Gerichtshof sah das anders und das Urteil trat dennoch in Kraft.

Daraufhin stellt Google ein Onlineformular zur Verf�gung, womit ein Jeder die M�glichkeit hat, einen Antrag auf L�schung von Google Inhalten und Suchergebnissen zu stellen. Bereits in den ersten f�nf Monaten erhielt Google circa 143000 L�schanfragen, das zur Konsequenz hatte, dass Google die Rechtsabteilung aufstocken musste. Denn die Aufgabe zu beurteilen, welche Daten unangemessen, irrelevant, exzessiv, nicht mehr aktuell oder nicht mehr relevant sind und nicht in Konflikt mit dem �ffentlichen Interesse stehen, liegt bei den Suchmaschinen.

\section{Kritik}
\textit{"`The right to be forgotten"'} bekam auch viel Kritik, dass es die Redefreiheit untergr�bt und eine Zensur des Internet ausl�st. Au�erdem kann es vorkommen, dass �ffentliche Interessen nicht mehr gewahrt werden und eventuell die M�glichkeit besteht, die Geschichte umzuschreiben. Ein weiterer gro�e Kritikpunkt war der Interpretationsspielraum den der Europ�ische Gerichtshof offen lie�, welche Resultate nun das Recht auf L�schung haben und welcher nicht. Es obliegt den Suchmaschinen Entscheidungen zu treffen, welche nicht einmal der Europ�ische Gerichtshof f�llen konnte oder wollte.

Das Urteil bringt eine gro�e Umstellung f�r Suchmaschinen mit sich, was kein Problem f�r einen Giganten, wie Google darstellt, welcher sich bereits zuvor mit dieser Materie auseinander setzen musste. Es d�rfte jedoch schwer f�r kleine Suchmaschinen umzusetzen sein. Zus�tzlich bringt das Gesetz kein wirkliches Vergessen, sondern die zu vergessenden Daten werden lediglich schwerer auffindbar.

\section{International}
International gesehen gibt es \textit{"`The right to be forgotten"'} noch in Argentinien, welche auch ein Gesetz f�r die L�schung von Suchresultaten von Suchmaschinen im Internet erlie�en. Nat�rlich gab es zu dem Urteil viel globales Aufsehen und weltweit gibt es auch einige Prozesse dazu, welche jedoch bis jetzt immer zu Gunsten des �ffentlichen Interesses endeten. Man denkt jedoch, dass es ein Thema f�r die Zukunft werden kann.

Man muss sich auch die Frage stellen, ob sich Herr Gonz�les der Ironie bewusst war, dass er bei Erfolg des Prozesses auf ewig daf�r in Erinnerung bleibt und auch die Informationen, welche er l�schen lassen wollte, noch mehr Aufmerksamkeit dadurch erhalten.