\documentclass{beamer}

\usepackage[ngerman]{babel}
\usepackage[utf8x]{inputenc}
\usepackage{graphicx}
%\usepackage{default}
\usetheme{Copenhagen}
\usecolortheme{sidebartab}

\title[The right to be forgotten]{The right to be forgotten}
\author[S. Vikoler]{Stefan Vikoler}
\institute[Computer Science - University of Salzburg]{Seminar aus Informatik\\ Computer Science\\ University of Salzburg}
\date[21.05.2015]{21. Mai 2015}

\begin{document}

\frame{
  \titlepage
}
\frame{
	\frametitle{Inhaltsverzeichnis}
	\tableofcontents[hideallsubsections]
}
\section{Hintergr\"unde}
\subsection{Einleitung}
\frame{
  \frametitle{Einleitung}
    \begin{itemize}
      \item Präzedenzfall
			\item Europäischer Gerichtshof
      \item 13. Mai 2014
      \item Google Link verletzt Privatsphäre
      \item Recht auf Löschung von Suchresultaten
			\item Balance zwischen privaten und öffentlichen Interessen
    \end{itemize}
}
\frame{
  \frametitle{Werdegang}
    \begin{itemize}
      \item Mario Costeja González
			\item Sozialversicherungsschulden im Jahr 1998
      \item zwei Zeitungsartikel von La Vanguardia Ediciones
			\item Klage gegen Zeitung
      \item Klage gegen Google
    \end{itemize}
}
\subsection{Urteil}
\frame{
  \frametitle{Urteil}
    \begin{itemize}
			\item Recht auf Löschung, wenn
			\begin{itemize}
				\item unangemessen
				\item irrelevant
				\item exzessiv
				\item nicht mehr aktuell
				\item nicht mehr relevant
      \end{itemize}
			\item alle Suchmaschinen betroffen
      \item Suchmaschinen als Data Controller
    \end{itemize}
}

\section{Reaktionen}
\subsection{Google}
\frame{
  \frametitle{Google}
    \begin{itemize}
      \item Daten lokalisieren und präsentieren
			\item Balkanisierung des Internet
      \item Onlineformular
      \item 143000 Löschanfragen in ersten 5 Monaten
			\item Rechtsabteilung aufgestockt
    \end{itemize}
}
\subsection{Krititk}
\frame{
  \frametitle{Kritik}
    \begin{itemize}
      \item Redefreiheit
			\item Zensur des Internet
      \item öffentliche Interessen
      \item Geschichte umschreiben
			\item Interpretationsspielraum
			\item schwer Umzusetzen für klein Suchmaschinen
			\item kein wirkliches Vergessen
    \end{itemize}
}
\subsection{International}
\frame{
  \frametitle{International}
    \begin{itemize}
      \item globales Aufsehen
			\item Right to be forgotten in Argentinien
      \item weltweite Fälle
      \item Thema in der Zukunft
    \end{itemize}
}
\section{Digitales Vergessen}
\subsection{Vergeben und Vergessen}
\frame{
  \frametitle{Vergeben und Vergessen}
    \begin{itemize}
      \item religiös
			\item kulturell
      \item vergebene Taten werden vergessen
      \item Neustart in neuer Umgebung
			\item digitale Sünden
    \end{itemize}
}
\frame{
  \frametitle{Digitale Identität}
    \begin{itemize}
      \item Möglichkeit sich neu zu erfinden
			\item mehrere Pseudonyme
      \item Social Media
			\begin{itemize}
				\item Verschmelzung von privaten und öffentlichen Leben
			\end{itemize}
			\item Internet als Lügennetz
    \end{itemize}
}
\subsection{M\"oglichkeiten}
\frame{
  \frametitle{Verbreitung kontrollieren}
    \begin{itemize}
      \item Originator-Controlled access control
			\item Links zu Daten verbreiten
      \item Information lifetime
      \item Wer besitzt Daten?
    \end{itemize}
}
\frame{
  \frametitle{Verstecken}
    \begin{itemize}
      \item Irreführen und überfluten
			\begin{itemize}
				\item 2. Weltkrieg
				\item Honeypots
			\end{itemize}
			\item Unterschieben
      \item Bedeutung verändern
    \end{itemize}
}

\section{Vergessliche Programme}
\subsection{Einleitung}
\frame{
  \frametitle{Einleitung}
    \begin{itemize}
      \item Policy
			\item Sensible Daten garantiert entfernen
      \item garanteed data lifetime property
      \item ohne Unterstützung der Applikation
			\item Virtual Machine
    \end{itemize}
}
\frame{
  \frametitle{Ziel}
    \begin{itemize}
      \item garanteed data lifetime
			\begin{itemize}
				\item ohne auf Applikation angewiesen zu sein
				\item ohne Beeinträchtigung der Applikation
				\item ohne Einschränkung der Benutzbarkeit
			\end{itemize}
    \end{itemize}
}
\frame{
  \frametitle{Related Work}
    \begin{itemize}
      \item sofortige Löschung nach Deallozierung von Daten
			\item sensible Daten werden in standalone VM bearbeitet
      \item Shadow processes
      \item File systems
    \end{itemize}
}
\subsection{State Reincarnation}
\frame{
  \frametitle{Framework}
    \begin{itemize}
      \item Snapshot bei Empfang von sensiblen Daten als Checkpoint
			\item Logging aller Inputs nach Empfang
      \item Replay vom Checkpoint
    \end{itemize}
}
\frame{
  \frametitle{3 Phasen}
    \begin{itemize}
      \item sensiblen Input erkennen
			\item Checkpoints und Logging
      \item Replay mit verändertem Input
    \end{itemize}
}
\subsection{Herausforderungen}
\frame{
  \frametitle{Fidelity}
    \begin{itemize}
      \item Applikations-Zustand nach Replay
			\item Replay with Omission
			\item Replay with Substitution
			\item Replay with Consistent Substitution
    \end{itemize}
}
\frame{
  \frametitle{Pervasiveness}
    \begin{itemize}
      \item sensible Daten die im System verweilen
			\item gespeicherte Kopien durch
			\begin{itemize}
				\item Betriebssystem
				\item Screen-Output
				\item Network-Output
			\end{itemize}
      \item Virtual Machine Monitor
      \begin{itemize}
				\item ohne Buffer
			\end{itemize}
    \end{itemize}
}
\frame{
  \frametitle{Containment}
    \begin{itemize}
      \item Kommunikation zwischen anderen Entitäten
			\item Outputs
			\begin{itemize}
				\item besitzt selbst Lifetime Framework
				\item warten bis lifetime ablauft
				\item Entscheidung des Benutzers
			\end{itemize}
			\item Inputs
			\begin{itemize}
				\item Problem für die Benutzbarkeit
				\item bekannte Requests und Responses versuchen abzufangen
				\item einfach ignorieren
				\item wird noch untersucht
			\end{itemize}
    \end{itemize}
}
\frame{
  \frametitle{Overhead}
    \begin{itemize}
      \item Performance-Probleme
			\item klein beim Erkennen der sensiblen Daten
      \item klein beim Erstellen des Checkpoints
      \item groß beim Logging
			\item Replay kann parallel laufen
    \end{itemize}
}
\frame{
  \includegraphics[width=300px]{ending.jpg}
}

\end{document}
